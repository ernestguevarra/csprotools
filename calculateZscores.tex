\documentclass[12pt,a4paper]{article}
\usepackage{lmodern}
\usepackage{amssymb,amsmath}
\usepackage{ifxetex,ifluatex}
\usepackage{fixltx2e} % provides \textsubscript
\ifnum 0\ifxetex 1\fi\ifluatex 1\fi=0 % if pdftex
  \usepackage[T1]{fontenc}
  \usepackage[utf8]{inputenc}
\else % if luatex or xelatex
  \ifxetex
    \usepackage{mathspec}
  \else
    \usepackage{fontspec}
  \fi
  \defaultfontfeatures{Ligatures=TeX,Scale=MatchLowercase}
\fi
% use upquote if available, for straight quotes in verbatim environments
\IfFileExists{upquote.sty}{\usepackage{upquote}}{}
% use microtype if available
\IfFileExists{microtype.sty}{%
\usepackage{microtype}
\UseMicrotypeSet[protrusion]{basicmath} % disable protrusion for tt fonts
}{}
\usepackage[margin=2cm]{geometry}
\usepackage{hyperref}
\PassOptionsToPackage{usenames,dvipsnames}{color} % color is loaded by hyperref
\hypersetup{unicode=true,
            pdftitle={Calculating anthropometric measurement flags in CSPro},
            colorlinks=true,
            linkcolor=blue,
            citecolor=blue,
            urlcolor=blue,
            breaklinks=true}
\urlstyle{same}  % don't use monospace font for urls
\usepackage{natbib}
\bibliographystyle{plainnat}
\usepackage{color}
\usepackage{fancyvrb}
\newcommand{\VerbBar}{|}
\newcommand{\VERB}{\Verb[commandchars=\\\{\}]}
\DefineVerbatimEnvironment{Highlighting}{Verbatim}{commandchars=\\\{\}}
% Add ',fontsize=\small' for more characters per line
\usepackage{framed}
\definecolor{shadecolor}{RGB}{248,248,248}
\newenvironment{Shaded}{\begin{snugshade}}{\end{snugshade}}
\newcommand{\KeywordTok}[1]{\textcolor[rgb]{0.13,0.29,0.53}{\textbf{#1}}}
\newcommand{\DataTypeTok}[1]{\textcolor[rgb]{0.13,0.29,0.53}{#1}}
\newcommand{\DecValTok}[1]{\textcolor[rgb]{0.00,0.00,0.81}{#1}}
\newcommand{\BaseNTok}[1]{\textcolor[rgb]{0.00,0.00,0.81}{#1}}
\newcommand{\FloatTok}[1]{\textcolor[rgb]{0.00,0.00,0.81}{#1}}
\newcommand{\ConstantTok}[1]{\textcolor[rgb]{0.00,0.00,0.00}{#1}}
\newcommand{\CharTok}[1]{\textcolor[rgb]{0.31,0.60,0.02}{#1}}
\newcommand{\SpecialCharTok}[1]{\textcolor[rgb]{0.00,0.00,0.00}{#1}}
\newcommand{\StringTok}[1]{\textcolor[rgb]{0.31,0.60,0.02}{#1}}
\newcommand{\VerbatimStringTok}[1]{\textcolor[rgb]{0.31,0.60,0.02}{#1}}
\newcommand{\SpecialStringTok}[1]{\textcolor[rgb]{0.31,0.60,0.02}{#1}}
\newcommand{\ImportTok}[1]{#1}
\newcommand{\CommentTok}[1]{\textcolor[rgb]{0.56,0.35,0.01}{\textit{#1}}}
\newcommand{\DocumentationTok}[1]{\textcolor[rgb]{0.56,0.35,0.01}{\textbf{\textit{#1}}}}
\newcommand{\AnnotationTok}[1]{\textcolor[rgb]{0.56,0.35,0.01}{\textbf{\textit{#1}}}}
\newcommand{\CommentVarTok}[1]{\textcolor[rgb]{0.56,0.35,0.01}{\textbf{\textit{#1}}}}
\newcommand{\OtherTok}[1]{\textcolor[rgb]{0.56,0.35,0.01}{#1}}
\newcommand{\FunctionTok}[1]{\textcolor[rgb]{0.00,0.00,0.00}{#1}}
\newcommand{\VariableTok}[1]{\textcolor[rgb]{0.00,0.00,0.00}{#1}}
\newcommand{\ControlFlowTok}[1]{\textcolor[rgb]{0.13,0.29,0.53}{\textbf{#1}}}
\newcommand{\OperatorTok}[1]{\textcolor[rgb]{0.81,0.36,0.00}{\textbf{#1}}}
\newcommand{\BuiltInTok}[1]{#1}
\newcommand{\ExtensionTok}[1]{#1}
\newcommand{\PreprocessorTok}[1]{\textcolor[rgb]{0.56,0.35,0.01}{\textit{#1}}}
\newcommand{\AttributeTok}[1]{\textcolor[rgb]{0.77,0.63,0.00}{#1}}
\newcommand{\RegionMarkerTok}[1]{#1}
\newcommand{\InformationTok}[1]{\textcolor[rgb]{0.56,0.35,0.01}{\textbf{\textit{#1}}}}
\newcommand{\WarningTok}[1]{\textcolor[rgb]{0.56,0.35,0.01}{\textbf{\textit{#1}}}}
\newcommand{\AlertTok}[1]{\textcolor[rgb]{0.94,0.16,0.16}{#1}}
\newcommand{\ErrorTok}[1]{\textcolor[rgb]{0.64,0.00,0.00}{\textbf{#1}}}
\newcommand{\NormalTok}[1]{#1}
\usepackage{longtable,booktabs}
\usepackage{graphicx,grffile}
\makeatletter
\def\maxwidth{\ifdim\Gin@nat@width>\linewidth\linewidth\else\Gin@nat@width\fi}
\def\maxheight{\ifdim\Gin@nat@height>\textheight\textheight\else\Gin@nat@height\fi}
\makeatother
% Scale images if necessary, so that they will not overflow the page
% margins by default, and it is still possible to overwrite the defaults
% using explicit options in \includegraphics[width, height, ...]{}
\setkeys{Gin}{width=\maxwidth,height=\maxheight,keepaspectratio}
\IfFileExists{parskip.sty}{%
\usepackage{parskip}
}{% else
\setlength{\parindent}{0pt}
\setlength{\parskip}{6pt plus 2pt minus 1pt}
}
\setlength{\emergencystretch}{3em}  % prevent overfull lines
\providecommand{\tightlist}{%
  \setlength{\itemsep}{0pt}\setlength{\parskip}{0pt}}
\setcounter{secnumdepth}{0}
% Redefines (sub)paragraphs to behave more like sections
\ifx\paragraph\undefined\else
\let\oldparagraph\paragraph
\renewcommand{\paragraph}[1]{\oldparagraph{#1}\mbox{}}
\fi
\ifx\subparagraph\undefined\else
\let\oldsubparagraph\subparagraph
\renewcommand{\subparagraph}[1]{\oldsubparagraph{#1}\mbox{}}
\fi

%%% Use protect on footnotes to avoid problems with footnotes in titles
\let\rmarkdownfootnote\footnote%
\def\footnote{\protect\rmarkdownfootnote}

%%% Change title format to be more compact
\usepackage{titling}

% Create subtitle command for use in maketitle
\newcommand{\subtitle}[1]{
  \posttitle{
    \begin{center}\large#1\end{center}
    }
}

\setlength{\droptitle}{-2em}

  \title{Calculating anthropometric measurement flags in CSPro}
    \pretitle{\vspace{\droptitle}\centering\huge}
  \posttitle{\par}
    \author{}
    \preauthor{}\postauthor{}
      \predate{\centering\large\emph}
  \postdate{\par}
    \date{20 July 2018}

\usepackage{float} 
\usepackage{setspace}
\usepackage[utf8]{inputenc}
\onehalfspacing
\usepackage{booktabs}
\usepackage{longtable}
\usepackage{array}
\usepackage{multirow}
\usepackage[table]{xcolor}
\usepackage{wrapfig}
\usepackage{float}
\usepackage{colortbl}
\usepackage{pdflscape}
\usepackage{tabu}
\usepackage{threeparttable}
\usepackage{threeparttablex}
\usepackage[normalem]{ulem}
\usepackage{makecell}

\usepackage{amsthm}
\newtheorem{theorem}{Theorem}
\newtheorem{lemma}{Lemma}
\theoremstyle{definition}
\newtheorem{definition}{Definition}
\newtheorem{corollary}{Corollary}
\newtheorem{proposition}{Proposition}
\theoremstyle{definition}
\newtheorem{example}{Example}
\theoremstyle{definition}
\newtheorem{exercise}{Exercise}
\theoremstyle{remark}
\newtheorem*{remark}{Remark}
\newtheorem*{solution}{Solution}
\begin{document}
\maketitle

\hypertarget{calculating-z-scores-within-cspro}{%
\section{Calculating z-scores within
CSPro}\label{calculating-z-scores-within-cspro}}

CSPro logic supports the following mathematical operators:

\rowcolors{2}{gray!6}{white}\begin{table}[H]

\caption{\label{tab:mathoperators}Mathematical operations supported by CSPro}
\centering
\fontsize{12}{14}\selectfont
\begin{tabular}[t]{rr}
\hiderowcolors
\toprule
\textbf{Operations} & \textbf{Operands}\\
\midrule
\showrowcolors
Addition & +\\
Subtraction & -\\
Multiplication & *\\
Division & /\\
Modulo & \%\\
Exponentiation & \textasciicircum{}\\
\bottomrule
\end{tabular}
\end{table}\rowcolors{2}{white}{white}

Using these operations, we can look into a way to flag weight and height
measurements in CSPro. We are trying out an approach developed by Robert
Johnston
(\href{mailto:rfjohnstonunicef@gmail.com}{\nolinkurl{rfjohnstonunicef@gmail.com}})\footnote{see
  \url{https://github.com/RobertJohnston/SMART-Questionnaire}} from
UNICEF that calculates the corresponding value for the anthropometric
measurement being standardised at a specific standard deviation which in
context of a flag would be set at the the upper and lower bounds used in
the flagging criteria. The WHO flagging criteria are used. Below we show
the formula used for these calculations.

\newpage

\hypertarget{calculate-lower-and-upper-height-measurement-given-a-childs-age-and-sex-that-will-produce-a-sd-above-and-below-6-sd-and--6-sd-respectively}{%
\subsection{Calculate lower and upper height measurement given a child's
age and sex that will produce a SD above and below +6 SD and -6 SD
respectively}\label{calculate-lower-and-upper-height-measurement-given-a-childs-age-and-sex-that-will-produce-a-sd-above-and-below-6-sd-and--6-sd-respectively}}

This is to calculate corresponding flags for height measurements given
the age and sex of the child. The idea here will be that the height will
be checked by calculating the height that would give an SD greater than
+6 or lower than -6 using the child's age and sex.

The general formula is:

\[\begin{aligned}
height_\text{lower for boys} ~ = ~ & 0.000000379386 ~ \times ~ age ^ 5 ~ - ~ 0.000069515524 ~ \times ~ age ^ 4 \\
& + ~ 0.004909805 ~ \times ~ age ^ 3 ~ - ~ 0.168908042 ~ \times ~ age ^ 2 \\
& + ~ 3.266127182 ~ \times ~ age ~ + ~ 39.95107423 \\
\\
height_\text{upper for boys} ~ = ~ & 0.000000477257 ~ \times ~ age ^ 5 ~ - ~ 0.000080053304 ~ \times ~ age ^ 4 \\
& + ~ 0.005071378 ~ \times ~ age ^ 3 ~ - ~ 0.158048367 ~ \times ~ age ^ 2 \\
& + ~ 3.584968875 ~ \times ~ age ~ + ~ 63.13173807 \\
\\
height_\text{lower for girls} ~ = ~ & 0.000000308918 ~ \times ~ age ^ 5 ~ - ~ 0.000056212972 ~ \times ~ age ^ 4 \\
& + ~ 0.003932757 ~ \times ~ age ^ 3 ~ - ~ 0.135180355 ~ \times ~ age ^ 2 \\
& + ~ 2.789665701 ~ \times ~ age ~ + ~ 39.23327346 \\
\\
height_\text{upper for girls} ~ = ~ & 0.000000371919 ~ \times ~ age ^ 5 ~ - ~ 0.000065134503 ~ \times ~ age ^ 4 \\
& + ~ 0.004358773 ~ \times ~ age ^ 3 ~ - ~ 0.145396571 ~ \times ~ age ^ 2 \\
& + ~ 3.553078292 ~ \times ~ age ~ + ~ 62.01230832 \\
\\
where: & \\
\\
age & ~ = ~ \text{age of child} \\
height_\text{lower for boys} & ~ = ~ \text{expected height for boys for a -6SD} \\
height_\text{upper for boys} & ~ = ~ \text{expected height for boys for a +6SD} \\
height_\text{lower for girls} & ~ = ~ \text{expected height for girls for a -6SD} \\
height_\text{upper for girls} & ~ = ~ \text{expected height for girls for a +6SD}
\end{aligned}\]

~

In CSPro, these calculations can be approached similarly. In the
following examples, we assume that the dataset captured in CSPro has the
following variables:

\begin{table}[H]
\centering\begingroup\fontsize{12}{14}\selectfont
\rowcolors{2}{gray!6}{white}

\begin{tabular}{ll}
\hiderowcolors
\toprule
\textbf{Variables} & \textbf{Definitions}\\
\midrule
\showrowcolors
cage & Age of child in months\\
csex & Sex of child\\
height & Height of child\\
\bottomrule
\end{tabular}
\rowcolors{2}{white}{white}\endgroup{}
\end{table}

~

The possible steps/code in CSPro for creating logic for flagging HAZ can
be:

\begin{Shaded}
\begin{Highlighting}[]
\OperatorTok{/}\ErrorTok{/}\StringTok{ }\NormalTok{Declare variables}
\NormalTok{numeric aNumber;}

\OperatorTok{/}\ErrorTok{/}\StringTok{ }\NormalTok{Create logic variables }\ControlFlowTok{for}\NormalTok{ each of the HAZ flags}
\NormalTok{numeric hazLowerBoys;}
\NormalTok{hazLowerBoys =}\StringTok{ }\FloatTok{0.000000379386} \OperatorTok{*}\StringTok{ }\NormalTok{cage }\OperatorTok{^}\StringTok{ }\DecValTok{5} \OperatorTok{-}\StringTok{ }\FloatTok{0.000069515524} \OperatorTok{*}\StringTok{ }\NormalTok{cage }\OperatorTok{^}\StringTok{ }\DecValTok{4} 
               \OperatorTok{+}\StringTok{ }\FloatTok{0.004909805} \OperatorTok{*}\StringTok{ }\NormalTok{cage }\OperatorTok{^}\StringTok{ }\DecValTok{3} \OperatorTok{-}\StringTok{ }\FloatTok{0.168908042} \OperatorTok{*}\StringTok{ }\NormalTok{cage }\OperatorTok{^}\StringTok{ }\DecValTok{2} 
               \OperatorTok{+}\StringTok{ }\FloatTok{3.266127182} \OperatorTok{*}\StringTok{ }\NormalTok{cage }\OperatorTok{+}\StringTok{ }\FloatTok{39.95107423}

\NormalTok{numeric hazUpperBoys;}
\NormalTok{hazUpperBoys =}\StringTok{ }\FloatTok{0.000000477257} \OperatorTok{*}\StringTok{ }\NormalTok{cage }\OperatorTok{^}\StringTok{ }\DecValTok{5} \OperatorTok{-}\StringTok{ }\FloatTok{0.000080053304} \OperatorTok{*}\StringTok{ }\NormalTok{cage }\OperatorTok{^}\StringTok{ }\DecValTok{4} 
               \OperatorTok{+}\StringTok{ }\FloatTok{0.005071378} \OperatorTok{*}\StringTok{ }\NormalTok{cage }\OperatorTok{^}\StringTok{ }\DecValTok{3} \OperatorTok{-}\StringTok{ }\FloatTok{0.158048367} \OperatorTok{*}\StringTok{ }\NormalTok{cage }\OperatorTok{^}\StringTok{ }\DecValTok{2} 
               \OperatorTok{+}\StringTok{ }\FloatTok{3.584968875} \OperatorTok{*}\StringTok{ }\NormalTok{cage }\OperatorTok{+}\StringTok{ }\FloatTok{63.13173807}

\NormalTok{numeric hazLowerGirls;}
\NormalTok{hazLowerGirls =}\StringTok{ }\FloatTok{0.000000308918} \OperatorTok{*}\StringTok{ }\NormalTok{cage }\OperatorTok{^}\StringTok{ }\DecValTok{5} \OperatorTok{-}\StringTok{ }\FloatTok{0.000056212972} \OperatorTok{*}\StringTok{ }\NormalTok{cage }\OperatorTok{^}\StringTok{ }\DecValTok{4}
                \OperatorTok{+}\StringTok{ }\FloatTok{0.003932757} \OperatorTok{*}\StringTok{ }\NormalTok{cage }\OperatorTok{^}\StringTok{ }\DecValTok{3} \OperatorTok{-}\StringTok{ }\FloatTok{0.135180355} \OperatorTok{*}\StringTok{ }\NormalTok{cage }\OperatorTok{^}\StringTok{ }\DecValTok{2}
                \OperatorTok{+}\StringTok{ }\FloatTok{2.789665701} \OperatorTok{*}\StringTok{ }\NormalTok{cage }\OperatorTok{+}\StringTok{ }\FloatTok{39.23327346}

\NormalTok{numeric hazUpperGirls;}
\NormalTok{hazUpperGirs =}\StringTok{ }\FloatTok{0.000000371919} \OperatorTok{*}\StringTok{ }\NormalTok{cage }\OperatorTok{^}\StringTok{ }\DecValTok{5} \OperatorTok{-}\StringTok{ }\FloatTok{0.000065134503} \OperatorTok{*}\StringTok{ }\NormalTok{cage }\OperatorTok{^}\StringTok{ }\DecValTok{4}
               \OperatorTok{+}\StringTok{ }\FloatTok{0.004358773} \OperatorTok{*}\StringTok{ }\NormalTok{cage }\OperatorTok{^}\StringTok{ }\DecValTok{3} \OperatorTok{-}\StringTok{ }\FloatTok{0.145396571} \OperatorTok{*}\StringTok{ }\NormalTok{cage }\OperatorTok{^}\StringTok{ }\DecValTok{2}
               \OperatorTok{+}\StringTok{ }\FloatTok{3.553078292} \OperatorTok{*}\StringTok{ }\NormalTok{cage }\OperatorTok{+}\StringTok{ }\FloatTok{62.01230832}
               
\OperatorTok{/}\ErrorTok{/}\StringTok{ }\NormalTok{Add logic to flag height measurements }\OperatorTok{-}\StringTok{ }\NormalTok{boys}
\ControlFlowTok{if}\NormalTok{ csex }\OperatorTok{<}\ErrorTok{>}\StringTok{ }\DecValTok{1}\NormalTok{ and height }\OperatorTok{>}\StringTok{ }\NormalTok{hazUpperBoys or height }\OperatorTok{<}\StringTok{ }\NormalTok{hazLowerBoys then}
  \KeywordTok{warning}\NormalTok{(}\StringTok{"Height measurement beyond expected value for child's age and sex"}\NormalTok{,}
\NormalTok{          height)}
          \KeywordTok{select}\NormalTok{(}\StringTok{"Repeat height measurement"}\NormalTok{, height,}
                 \StringTok{"Ignore warning"}\NormalTok{, continue);}
\NormalTok{endif;}

\OperatorTok{/}\ErrorTok{/}\StringTok{ }\NormalTok{Add logic to flag height measurements }\OperatorTok{-}\StringTok{ }\NormalTok{girls}
\ControlFlowTok{if}\NormalTok{ csex }\OperatorTok{<}\ErrorTok{>}\StringTok{ }\DecValTok{2}\NormalTok{ and height }\OperatorTok{>}\StringTok{ }\NormalTok{hazUpperGirls or height }\OperatorTok{<}\StringTok{ }\NormalTok{hazLowerGirls then}
  \KeywordTok{warning}\NormalTok{(}\StringTok{"Height measurement beyond expected value for child's age and sex"}\NormalTok{,}
\NormalTok{          height)}
          \KeywordTok{select}\NormalTok{(}\StringTok{"Repeat height measurement"}\NormalTok{, height,}
                 \StringTok{"Ignore warning"}\NormalTok{, continue);}
\NormalTok{endif;}
\end{Highlighting}
\end{Shaded}

\newpage

\hypertarget{calculate-lower-and-upper-weight-measurement-given-a-childs-age-and-sex-that-will-produce-a-sd-above-and-below-5-sd-and--6-sd-respectively}{%
\subsection{Calculate lower and upper weight measurement given a child's
age and sex that will produce a SD above and below +5 SD and -6 SD
respectively}\label{calculate-lower-and-upper-weight-measurement-given-a-childs-age-and-sex-that-will-produce-a-sd-above-and-below-5-sd-and--6-sd-respectively}}

This is to calculate corresponding flags for weight measurements given
the age and sex of the child. The idea here will be that the weight will
be checked by calculating the weight that would give an SD greater than
+5 or lower than -6 using the child's age and sex.

The general formula is:

\[\begin{aligned}
weight_\text{lower for boys} ~ = ~ & 0.000000095420 ~ \times ~ age ^ 5 ~ - ~ 0.00001662831 ~ \times ~ age ^ 4 \\
& + ~ 0.001091416 ~ \times ~ age ^ 3 ~ - ~ 0.033880085 ~ \times ~ age ^ 2 \\
& + ~ 0.562601613 ~ \times ~ age ~ + ~ 1.139474257 \\
\\
weight_\text{upper for boys} ~ = ~ &  0.000000234816 ~ \times ~ age ^ 5 ~ - ~ 0.000039531734 ~ \times ~ age ^ 4 \\
& + ~ 0.002496579 ~ \times ~ age ^ 3 ~ - ~ 0.072781678 ~ \times ~ age ^ 2 \\
& + ~ 1.351053186 ~ \times ~ age ~ + ~ 6.972621013 \\
\\
weight_\text{lower for girls} ~ = ~ & 0.000000083218 ~ \times ~ age ^ 5 ~ - ~ 0.000013843176 ~ \times ~ age ^ 4 \\
& + ~ 0.000861019 ~ \times ~ age ^ 3 ~ - ~ 0.025646557 ~ \times ~ age ^ 2 \\
& + ~ 0.450277222 ~ \times ~ age ~ + ~ 1.111128346 \\
\\
weight_\text{upper for girls} ~ = ~ & 0.000000197736 ~ \times ~ age ^ 5 ~ - ~ 0.000035478810 ~ \times ~ age ^ 4 \\
& + ~ 0.002404164 ~ \times ~ age ^ 3 ~ - ~ 0.074112047 ~ \times ~ age ^ 2 \\
& + ~ 1.434496211 ~ \times ~ age ~ + ~ 6.482817802 \\
\\
where: & \\
\\
age & ~ = ~ \text{age of child} \\
weight_\text{lower for boys} & ~ = ~ \text{expected weight for boys for a -6SD} \\
weight_\text{upper for boys} & ~ = ~ \text{expected weight for boys for a +5SD} \\
weight_\text{lower for girls} & ~ = ~ \text{expected weight for girls for a -6SD} \\
weight_\text{upper for girls} & ~ = ~ \text{expected weight for girls for a +5SD}
\end{aligned}\]

~

In CSPro, these calculations can be approached similarly. In the
following examples, we assume that the dataset captured in CSPro has the
following variables:

\begin{table}[H]
\centering\begingroup\fontsize{12}{14}\selectfont
\rowcolors{2}{gray!6}{white}

\begin{tabular}{ll}
\hiderowcolors
\toprule
\textbf{Variables} & \textbf{Definitions}\\
\midrule
\showrowcolors
cage & Age of child in months\\
csex & Sex of child\\
weight & Weight of child\\
\bottomrule
\end{tabular}
\rowcolors{2}{white}{white}\endgroup{}
\end{table}

~

The possible steps/code in CSPro for creating logic for flagging WAZ can
be:

\begin{Shaded}
\begin{Highlighting}[]
\OperatorTok{/}\ErrorTok{/}\StringTok{ }\NormalTok{Declare variables}
\NormalTok{numeric aNumber;}

\OperatorTok{/}\ErrorTok{/}\StringTok{ }\NormalTok{Create logic variables }\ControlFlowTok{for}\NormalTok{ each of the WAZ flags}
\NormalTok{numeric wazLowerBoys;}
\NormalTok{wazLowerBoys =}\StringTok{ }\FloatTok{0.000000095420} \OperatorTok{*}\StringTok{ }\NormalTok{cage }\OperatorTok{^}\StringTok{ }\DecValTok{5} \OperatorTok{-}\StringTok{ }\FloatTok{0.00001662831} \OperatorTok{*}\StringTok{ }\NormalTok{cage }\OperatorTok{^}\StringTok{ }\DecValTok{4} 
               \OperatorTok{+}\StringTok{ }\FloatTok{0.001091416} \OperatorTok{*}\StringTok{ }\NormalTok{cage }\OperatorTok{^}\StringTok{ }\DecValTok{3} \OperatorTok{-}\StringTok{ }\FloatTok{0.033880085} \OperatorTok{*}\StringTok{ }\NormalTok{cage }\OperatorTok{^}\StringTok{ }\DecValTok{2} 
               \OperatorTok{+}\StringTok{ }\FloatTok{0.562601613} \OperatorTok{*}\StringTok{ }\NormalTok{cage }\OperatorTok{+}\StringTok{ }\FloatTok{1.139474257}

\NormalTok{numeric wazUpperBoys;}
\NormalTok{wazUpperBoys =}\StringTok{ }\FloatTok{0.000000234816} \OperatorTok{*}\StringTok{ }\NormalTok{age }\OperatorTok{^}\StringTok{ }\DecValTok{5} \OperatorTok{-}\StringTok{ }\FloatTok{0.000039531734} \OperatorTok{*}\StringTok{ }\NormalTok{age }\OperatorTok{^}\StringTok{ }\DecValTok{4}
               \OperatorTok{+}\StringTok{ }\FloatTok{0.002496579} \OperatorTok{*}\StringTok{ }\NormalTok{age }\OperatorTok{^}\StringTok{ }\DecValTok{3} \OperatorTok{-}\StringTok{ }\FloatTok{0.072781678} \OperatorTok{*}\StringTok{ }\NormalTok{age }\OperatorTok{^}\StringTok{ }\DecValTok{2}
               \OperatorTok{+}\StringTok{ }\FloatTok{1.351053186} \OperatorTok{*}\StringTok{ }\NormalTok{age }\OperatorTok{+}\StringTok{ }\FloatTok{6.972621013}

\NormalTok{numeric wazLowerGirls;}
\NormalTok{wazLowerGirls =}\StringTok{ }\FloatTok{0.000000083218} \OperatorTok{*}\StringTok{ }\NormalTok{age }\OperatorTok{^}\StringTok{ }\DecValTok{5} \OperatorTok{-}\StringTok{ }\FloatTok{0.000013843176} \OperatorTok{*}\StringTok{ }\NormalTok{age }\OperatorTok{^}\StringTok{ }\DecValTok{4}
                \OperatorTok{+}\StringTok{ }\FloatTok{0.000861019} \OperatorTok{*}\StringTok{ }\NormalTok{age }\OperatorTok{^}\StringTok{ }\DecValTok{3} \OperatorTok{-}\StringTok{ }\FloatTok{0.025646557} \OperatorTok{*}\StringTok{ }\NormalTok{age }\OperatorTok{^}\StringTok{ }\DecValTok{2}
                \OperatorTok{+}\StringTok{ }\FloatTok{0.450277222} \OperatorTok{*}\StringTok{ }\NormalTok{age }\OperatorTok{+}\StringTok{ }\FloatTok{1.111128346}

\NormalTok{numeric wazUpperGirls;}
\NormalTok{wazUpperGirls =}\StringTok{ }\FloatTok{0.000000197736} \OperatorTok{*}\StringTok{ }\NormalTok{age }\OperatorTok{^}\StringTok{ }\DecValTok{5} \OperatorTok{-}\StringTok{ }\FloatTok{0.000035478810} \OperatorTok{*}\StringTok{ }\NormalTok{age }\OperatorTok{^}\StringTok{ }\DecValTok{4}
               \OperatorTok{+}\StringTok{ }\FloatTok{0.002404164} \OperatorTok{*}\StringTok{ }\NormalTok{age }\OperatorTok{^}\StringTok{ }\DecValTok{3} \OperatorTok{-}\StringTok{ }\FloatTok{0.074112047} \OperatorTok{*}\StringTok{ }\NormalTok{age }\OperatorTok{^}\StringTok{ }\DecValTok{2}
               \OperatorTok{+}\StringTok{ }\FloatTok{1.434496211} \OperatorTok{*}\StringTok{ }\NormalTok{age }\OperatorTok{+}\StringTok{ }\FloatTok{6.482817802}
               
\OperatorTok{/}\ErrorTok{/}\StringTok{ }\NormalTok{Add logic to flag weight measurements }\OperatorTok{-}\StringTok{ }\NormalTok{boys}
\ControlFlowTok{if}\NormalTok{ csex }\OperatorTok{<}\ErrorTok{>}\StringTok{ }\DecValTok{1}\NormalTok{ and weight }\OperatorTok{>}\StringTok{ }\NormalTok{wazUpperBoys or weight }\OperatorTok{<}\StringTok{ }\NormalTok{wazLowerBoys then}
  \KeywordTok{warning}\NormalTok{(}\StringTok{"Weight measurement beyond expected value for child's age and sex"}\NormalTok{,}
\NormalTok{          weight)}
          \KeywordTok{select}\NormalTok{(}\StringTok{"Repeat weight measurement"}\NormalTok{, weight,}
                 \StringTok{"Ignore warning"}\NormalTok{, continue);}
\NormalTok{endif;}

\OperatorTok{/}\ErrorTok{/}\StringTok{ }\NormalTok{Add logic to flag weight measurements }\OperatorTok{-}\StringTok{ }\NormalTok{girls}
\ControlFlowTok{if}\NormalTok{ csex }\OperatorTok{<}\ErrorTok{>}\StringTok{ }\DecValTok{2}\NormalTok{ and weight }\OperatorTok{>}\StringTok{ }\NormalTok{wazUpperGirls or weight }\OperatorTok{<}\StringTok{ }\NormalTok{wazLowerGirls then}
  \KeywordTok{warning}\NormalTok{(}\StringTok{"Weight measurement beyond expected value for child's age and sex"}\NormalTok{,}
\NormalTok{          weight)}
          \KeywordTok{select}\NormalTok{(}\StringTok{"Repeat weight measurement"}\NormalTok{, weight,}
                 \StringTok{"Ignore warning"}\NormalTok{, continue);}
\NormalTok{endif;}
\end{Highlighting}
\end{Shaded}

\newpage

\hypertarget{calculate-lower-and-upper-height-measurement-given-a-childs-weight-and-sex-that-will-produce-a-sd-above-and-below-5-sd-and--5-sd-respectively}{%
\subsection{Calculate lower and upper height measurement given a child's
weight and sex that will produce a SD above and below +5 SD and -5 SD
respectively}\label{calculate-lower-and-upper-height-measurement-given-a-childs-weight-and-sex-that-will-produce-a-sd-above-and-below-5-sd-and--5-sd-respectively}}

This is to calculate corresponding flags for height measurements given
the weight and sex of the child. The idea here will be that the height
will be checked by calculating the height that would give an SD greater
than +5 or lower than -5 using the child's weight and sex.

The general formula is:

\[\begin{aligned}
whz_\text{lower for boys} ~ = ~ & 0.002568778 ~ \times ~ weight ^ 5 ~ - ~ 0.087078285 ~ \times ~ weight ^ 4 \\
& + ~ 1.083870039 ~ \times ~ weight ^ 3 ~ - ~ 6.017158294 ~ \times ~ weight ^ 2 \\
& + ~ 20.69094143 ~ \times ~ weight ~ + ~ 24.23997191 \\
\\
whz_\text{upper for boys} ~ = ~ &  0.000039423 ~ \times ~ weight ^ 5 ~ - ~ 0.003300406 ~ \times ~ weight ^ 4 \\
& + ~ 0.100344392 ~ \times ~ weight ^ 3 ~ - ~ 1.359686971 ~ \times ~ weight ^ 2 \\
& + ~ 10.87955385 ~ \times ~ weight ~ + ~ 18.21716746 \\
\\
whz_\text{lower for girls} ~ = ~ & 0.001848563 ~ \times ~ weight ^ 5 ~ - ~ 0.0606399 ~ \times ~ weight ^ 4 \\
& + ~ 0.7185497 ~ \times ~ weight ^ 3 ~ - ~ 3.7764632 ~ \times ~ weight ^ 2 \\
& + ~ 15.4720170 ~ \times ~ weight ~ + ~ 28.0948931 \\
\\
whz_\text{upper for girls} ~ = ~ & 0.00002434 ~ \times ~ weight ^ 5 ~ - ~ 0.00197858 ~ \times ~ weight ^ 4 \\
& + ~ 0.05716011 ~ \times ~ weight ^ 3 ~ - ~ 0.71815707 ~ \times ~ weight ^ 2 \\
& + ~ 6.61322135 ~ \times ~ weight ~ + ~ 27.77925292 \\
\\
where: & \\
\\
weight & ~ = ~ \text{weight of child} \\
whz_\text{lower for boys} & ~ = ~ \text{expected height for boys for a -5SD} \\
whz_\text{upper for boys} & ~ = ~ \text{expected height for boys for a +5SD} \\
whz_\text{lower for girls} & ~ = ~ \text{expected height for girls for a -5SD} \\
whz_\text{upper for girls} & ~ = ~ \text{expected height for girls for a +5SD}
\end{aligned}\]

~

In CSPro, these calculations can be approached similarly. In the
following examples, we assume that the dataset captured in CSPro has the
following variables:

\begin{table}[H]
\centering\begingroup\fontsize{12}{14}\selectfont
\rowcolors{2}{gray!6}{white}

\begin{tabular}{ll}
\hiderowcolors
\toprule
\textbf{Variables} & \textbf{Definitions}\\
\midrule
\showrowcolors
csex & Sex of child\\
weight & Weight of child\\
height & Height of child\\
\bottomrule
\end{tabular}
\rowcolors{2}{white}{white}\endgroup{}
\end{table}

~

The possible steps/code in CSPro for creating logic for flagging WHZ can
be:

\begin{Shaded}
\begin{Highlighting}[]
\OperatorTok{/}\ErrorTok{/}\StringTok{ }\NormalTok{Declare variables}
\NormalTok{numeric aNumber;}

\OperatorTok{/}\ErrorTok{/}\StringTok{ }\NormalTok{Create logic variables }\ControlFlowTok{for}\NormalTok{ each of the WAZ flags}
\NormalTok{numeric whzLowerBoys;}
\NormalTok{whzLowerBoys =}\StringTok{ }\FloatTok{0.002568778} \OperatorTok{*}\StringTok{ }\NormalTok{weight }\OperatorTok{^}\StringTok{ }\DecValTok{5} \OperatorTok{-}\StringTok{ }\FloatTok{0.087078285} \OperatorTok{*}\StringTok{ }\NormalTok{weight }\OperatorTok{^}\StringTok{ }\DecValTok{4}
               \OperatorTok{+}\StringTok{ }\FloatTok{1.083870039} \OperatorTok{*}\StringTok{ }\NormalTok{weight }\OperatorTok{^}\StringTok{ }\DecValTok{3} \OperatorTok{-}\StringTok{ }\FloatTok{6.017158294} \OperatorTok{*}\StringTok{ }\NormalTok{weight }\OperatorTok{^}\StringTok{ }\DecValTok{2}
               \OperatorTok{+}\StringTok{ }\FloatTok{20.69094143} \OperatorTok{*}\StringTok{ }\NormalTok{weight }\OperatorTok{+}\StringTok{ }\FloatTok{24.23997191}

\NormalTok{numeric whzUpperBoys;}
\NormalTok{whzUpperBoys =}\StringTok{ }\FloatTok{0.000039423} \OperatorTok{*}\StringTok{ }\NormalTok{weight }\OperatorTok{^}\StringTok{ }\DecValTok{5} \OperatorTok{-}\StringTok{ }\FloatTok{0.003300406} \OperatorTok{*}\StringTok{ }\NormalTok{weight }\OperatorTok{^}\StringTok{ }\DecValTok{4}
               \OperatorTok{+}\StringTok{ }\FloatTok{0.100344392} \OperatorTok{*}\StringTok{ }\NormalTok{weight }\OperatorTok{^}\StringTok{ }\DecValTok{3} \OperatorTok{-}\StringTok{ }\FloatTok{1.359686971} \OperatorTok{*}\StringTok{ }\NormalTok{weight }\OperatorTok{^}\StringTok{ }\DecValTok{2}
               \OperatorTok{+}\StringTok{ }\FloatTok{10.87955385} \OperatorTok{*}\StringTok{ }\NormalTok{weight }\OperatorTok{+}\StringTok{ }\FloatTok{18.21716746}

\NormalTok{numeric whzLowerGirls;}
\NormalTok{whzLowerGirls =}\StringTok{ }\FloatTok{0.001848563} \OperatorTok{*}\StringTok{ }\NormalTok{weight }\OperatorTok{^}\StringTok{ }\DecValTok{5} \OperatorTok{-}\StringTok{ }\FloatTok{0.0606399} \OperatorTok{*}\StringTok{ }\NormalTok{weight }\OperatorTok{^}\StringTok{ }\DecValTok{4}
                \OperatorTok{+}\StringTok{ }\FloatTok{0.7185497} \OperatorTok{*}\StringTok{ }\NormalTok{weight }\OperatorTok{^}\StringTok{ }\DecValTok{3} \OperatorTok{-}\StringTok{ }\FloatTok{3.7764632} \OperatorTok{*}\StringTok{ }\NormalTok{weight }\OperatorTok{^}\StringTok{ }\DecValTok{2}
                \OperatorTok{+}\StringTok{ }\FloatTok{15.4720170} \OperatorTok{*}\StringTok{ }\NormalTok{weight }\OperatorTok{+}\StringTok{ }\FloatTok{28.0948931}

\NormalTok{numeric whzUpperGirls;}
\NormalTok{whzUpperGirls =}\StringTok{ }\FloatTok{0.00002434} \OperatorTok{*}\StringTok{ }\NormalTok{weight }\OperatorTok{^}\StringTok{ }\DecValTok{5} \OperatorTok{-}\StringTok{ }\FloatTok{0.00197858} \OperatorTok{*}\StringTok{ }\NormalTok{weight }\OperatorTok{^}\StringTok{ }\DecValTok{4}
                \OperatorTok{+}\StringTok{ }\FloatTok{0.05716011} \OperatorTok{*}\StringTok{ }\NormalTok{weight }\OperatorTok{^}\StringTok{ }\DecValTok{3} \OperatorTok{-}\StringTok{ }\FloatTok{0.71815707} \OperatorTok{*}\StringTok{ }\NormalTok{weight }\OperatorTok{^}\StringTok{ }\DecValTok{2}
                \OperatorTok{+}\StringTok{ }\FloatTok{6.61322135} \OperatorTok{*}\StringTok{ }\NormalTok{weight }\OperatorTok{+}\StringTok{ }\FloatTok{27.77925292}
               
\OperatorTok{/}\ErrorTok{/}\StringTok{ }\NormalTok{Add logic to flag height measurements }\OperatorTok{-}\StringTok{ }\NormalTok{boys}
\ControlFlowTok{if}\NormalTok{ csex }\OperatorTok{<}\ErrorTok{>}\StringTok{ }\DecValTok{1}\NormalTok{ and height }\OperatorTok{>}\StringTok{ }\NormalTok{whzUpperBoys or height }\OperatorTok{<}\StringTok{ }\NormalTok{whzLowerBoys then}
  \KeywordTok{warning}\NormalTok{(}\StringTok{"Height measurement beyond expected value for child's weight and sex"}\NormalTok{,}
\NormalTok{          height)}
          \KeywordTok{select}\NormalTok{(}\StringTok{"Repeat height measurement"}\NormalTok{, height,}
                 \StringTok{"Ignore warning"}\NormalTok{, continue);}
\NormalTok{endif;}

\OperatorTok{/}\ErrorTok{/}\StringTok{ }\NormalTok{Add logic to flag height measurements }\OperatorTok{-}\StringTok{ }\NormalTok{girls}
\ControlFlowTok{if}\NormalTok{ csex }\OperatorTok{<}\ErrorTok{>}\StringTok{ }\DecValTok{2}\NormalTok{ and height }\OperatorTok{>}\StringTok{ }\NormalTok{whzUpperGirls or height }\OperatorTok{<}\StringTok{ }\NormalTok{whzLowerGirls then}
  \KeywordTok{warning}\NormalTok{(}\StringTok{"Height measurement beyond expected value for child's age and sex"}\NormalTok{,}
\NormalTok{          height)}
          \KeywordTok{select}\NormalTok{(}\StringTok{"Repeat height measurement"}\NormalTok{, height,}
                 \StringTok{"Ignore warning"}\NormalTok{, continue);}
\NormalTok{endif;}
\end{Highlighting}
\end{Shaded}

\newpage

\hypertarget{determining-sam-children-by-whz}{%
\subsection{Determining SAM children by
WHZ}\label{determining-sam-children-by-whz}}

This approach can also be used to identify children as SAM by WHZ by
calculating height of child given weight that will give an SD of -3.

The general formula is:

\[\begin{aligned}
whz_\text{-3SD boys} ~ = ~ & 0.000995821 ~ \times ~ weight ^ 5 ~ - ~ 0.041179737 ~ \times ~ weight ^ 4 \\
& + ~ 0.622839074 ~ \times ~ weight ^ 3 ~ - ~ 4.192765025 ~ \times ~ weight ^ 2 \\
& + ~ 17.49551065 ~ \times ~ weight ~ + ~ 23.63868321 \\
\\
whz_\text{-3SD for girls} ~ = ~ &  0.000754341 ~ \times ~ weight ^ 5 ~ - ~ 0.0299748 ~ \times ~ weight ^ 4 \\
& + ~ 0.4284523 ~ \times ~ weight ^ 3 ~ - ~ 2.7036671 ~ \times ~ weight ^ 2 \\
& + ~ 13.0452680 ~ \times ~ weight ~ + ~ 27.9253576 \\
\\
where: & \\
\\
weight & ~ = ~ \text{weight of child} \\
whz_\text{-3SD boys} & ~ = ~ \text{expected height for boys for a -3SD} \\
whz_\text{-3SD girls} & ~ = ~ \text{expected height for girls for a -3SD}
\end{aligned}\]


\end{document}
